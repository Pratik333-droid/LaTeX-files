\documentclass[12pt]{article}
\usepackage{amsmath, amssymb}
\begin{document}
\section{Harmonic Function}
A real valued function $h = \phi (x,y)$ of two variables is said to be \textit{Harmonic function} in a certain domain of xy-plane it it has continuous partial derivatives of the first and second order and satisfies the Laplace's equation $\dfrac{\partial^2\phi}{\partial x^2} + \dfrac{\partial^2\phi}{\partial y^2} = 0\\$

\noindent
\textit{Theorem 1: \\ If a function f(z) = u(x,y) + iv(x,y) is analytic in a domain D, then its component functions u and v are harmonic in D.\\Proof:}

Given that $f(z)$ is analytic so that two functions u(x,y) and v(x,y) have continuous partial derivatives of all orfers at that point and satisfying the relations

\begin{equation}
\dfrac{\partial u}{\partial x} = \dfrac{\partial v}{\partial y}
\end{equation}
and
\begin{equation}
\dfrac{\partial u}{\partial y} = -\dfrac{\partial v}{\partial x}
\end{equation}
Differentiating eq$^n$ (1) partially with respect to x, we get
\begin{equation}
\dfrac{\partial^2 u}{\partial x^2} = -\dfrac{\partial^2 v}{\partial x \ \partial y}
\end{equation}
Now,

\begin{equation*}
\dfrac{\partial^2 u}{\partial x^2} + \dfrac{\partial^2 u}{\partial y^2} = \dfrac{\partial^2 v}{\partial x \ \partial y} -\dfrac{\partial^2 v}{\partial y \ \partial x}
\end{equation*}
Since v is continuous, partial derivatives exists. Hence,
\begin{equation*}
\dfrac{\partial^2 v}{\partial x \ \partial y} = \dfrac{\partial^2 v}{\partial y \ \partial x}
\end{equation*}
\begin{equation*}
\dfrac{\partial^2 u}{\partial x^2} + \dfrac{\partial^2 u}{\partial y^2} = 0
\end{equation*}
\end{document}
